\documentclass[table]{beamer}

% Alternative to \usetheme{claw}
\usepackage{template/beamercolorthemeclaw}
\usepackage{template/beamerfontthemeclaw}
\usepackage{template/beamerinnerthemeclaw}
\usepackage{template/beamerouterthemeclaw}

\usepackage{standalone}
\usepackage{booktabs}
%\usepackage{hhline}
\usepackage{lmodern}


\definecolor{tab20darkblue}{HTML}{4e79a7}
\definecolor{tab20darkgreen}{HTML}{59a14f}
\definecolor{tab20darkred}{HTML}{e15759}
\definecolor{tab20darkorange}{HTML}{f28e2b}
\definecolor{tab20darkturquoise}{HTML}{499894}
\definecolor{tab20darkgray}{HTML}{79706e}
\definecolor{tab20darkbrown}{HTML}{9d7660}
\definecolor{tab20darkpurple}{HTML}{b07aa1}
\definecolor{tab20darkpink}{HTML}{e377c2}
\definecolor{tab20lightgreen}{HTML}{8cd17d}
\definecolor{tab20lightblue}{HTML}{a0cbd8}
\definecolor{tab20lightorange}{HTML}{ffbb78}
\definecolor{tab20lightred}{HTML}{ff9896}
\definecolor{tab20lightpurple}{HTML}{c5b0d5}
\definecolor{tab20lightbrown}{HTML}{c49c94}
\definecolor{tab20lightpink}{HTML}{f7b6d2}
\definecolor{tab20lightgray}{HTML}{c7c7c7}
\definecolor{tab20lightturquoise}{HTML}{9edae5}
\definecolor{tab20lightyellow}{HTML}{dbdb8d}
\definecolor{tab20darkyellow}{HTML}{bcbd22}

\definecolor{brass}{HTML}{E1C16E}



\input{packages.tex}

% Figures.
\usepackage{tikz-qtree}
\usepackage{pgfplots}
\pgfplotsset{compat=1.16}
\usetikzlibrary{shapes, positioning, arrows}



% 1st used for slides footer, 2nd for title slide & PDF metadata
\title[CausalQA]{Answering Causal Questions With Reinforcement Learning}
% Used for title slide & PDF metadata
\subtitle{}

%\date{April 27, 2022}
% Date with conference name
\date[\today]

% One author
% 1st used for slides footer, 2nd for title slide & PDF metadata
\author[Lukas Blübaum]{Lukas Blübaum \\ \vspace{0.5cm}Supervisor: Dr. Stefan Heindorf}

% 1st used for slides footer, 2nd for title slide
%\institute[DICE]{DICE Group\\Paderborn University}
% Institute with logo
\institute[DICE]{\includegraphics[height=0.6cm]{DICE}\\Data Science Group\\Paderborn University}

% Two authors from different institutes
% Source: https://www.overleaf.com/learn/latex/Beamer#The_title_page
%\author[\underline{Wilke}, Ngonga]{\underline{Adrian~Wilke}\inst{1} \and Axel~Ngonga\inst{2}}
%\institute[]
%{
%	\inst{1}%
%	DICE Group\\
%	Department of Computer Science\\
%	Paderborn University, Germany
%	\and
%	\inst{2}%
%	DICE Group\\
%	Department of Computer Science\\
%	Paderborn University, Germany
%}

% One logo on title page
%\titlegraphic{\includegraphics[height=1cm]{DICE}}

% Two logos on title page
%\titlegraphic{
%	\includegraphics[height=1cm]{UPB}%
%	\hspace*{2cm}~%
%	\includegraphics[height=1cm]{DICE}%
%}

% Siavoosh Payandeh Azad Jan. 2019
\RequirePackage{etoolbox}
\RequirePackage{pgf} % for calculating the values for gradient
\RequirePackage{xcolor} % enables the use of cellcolor make sure you have [table] option in the document class 
% Color set related!
%\definecolor{high}{HTML}{ef3b2c}  % the color for the highest number in your data set
%\definecolor{high}{HTML}{b9f3be}
\definecolor{high}{HTML}{8cd17d}
%\definecolor{low}{HTML}{ef3b2c}
%\definecolor{low}{HTML}{ff9896}
%\definecolor{low}{HTML}{f79e4c}
%\definecolor{low}{HTML}{fbd769}
\definecolor{low}{HTML}{faf2a7}
%\definecolor{low}{HTML}{fff7bc}  % the color for the lowest number in your data set
\newcommand*{\opacity}{90}% here you can change the opacity of the background color!
%======================================
% Data set related!
\newcommand*{\minval}{0.38}% define the minimum value on your data set
\newcommand*{\maxval}{1.0}% define the maximum value in your data set!
%======================================
% gradient function!
\newcommand{\g}[1]{%
    % The values are calculated linearly between \minval and \maxval
    \ifdimcomp{#1pt}{>}{\maxval pt}{#1}{%
    \ifdimcomp{#1pt}{<}{\minval pt}{#1}{%
         \pgfmathparse{int(round(100*(#1/(\maxval-\minval))-(\minval*(100/(\maxval-\minval)))))}%
        \xdef\tempa{\pgfmathresult}%
        \cellcolor{high!\tempa!low!\opacity}%
    }}%
 }%
%======================================
% gradient function single cell! 
\newcommand{\gradientcell}[6]{
    % The values are calculated linearly between \minval and \maxval
    \ifdimcomp{#1pt}{>}{#3 pt}{#1}{
    \ifdimcomp{#1pt}{<}{#2 pt}{#1}{
         \pgfmathparse{int(round(100*(#1/(#3-#2))-(\minval*(100/(#3-#2)))))}
        \xdef\tempa{\pgfmathresult}
        \cellcolor{#5!\tempa!#4!#6} #1
    }}
 }

\pdfsuppresswarningpagegroup=1

\begin{document}

% Title page without background
%{\usebackgroundtemplate{}
%\frame[plain]{\titlepage}}

% Title page with background
{\usebackgroundtemplate{\includegraphics[width=\paperwidth]{background-title-upb}
% German UPB logo
%\usebackgroundtemplate{\includegraphics[width=\paperwidth]{background-title-upb-german}
}
\frame[plain]{\titlepage}}

% Begin counting at second frame
\addtocounter{framenumber}{-1}


\begin{frame}{Causal Question Answering}
  \begin{columns}[T]
    \column{0.62\textwidth}
  \begin{itemize}
    \item<1-> Causal Questions
    \begin{itemize}
      \item Determine relationships between causes and effects
      \item E.g. understand what effects a cause could have in the future
    \end{itemize}
    \item<2-> Use Cases
    \begin{itemize}
      \item Search engines
      \item Virtual assistants like Alexa
      \item Automated decision-making
    \end{itemize}
    \item<3-> Limitations of prior approaches
    \begin{itemize}
      \item Often \emph{not explainable}
      \item \emph{Lack of large-scale datasets} of causal relations of high quality
    \end{itemize}
  \end{itemize}
  \column{0.38\textwidth}
  
  \scriptsize
  \emph{\textbf{Binary Causal Questions}}\\
  \emph{Does pneumonia cause anemia?}\\
  \vspace{0.2cm}
  \textbf{Open-Ended Questions}\\
  What can cause anemia?\\
  \vspace{0.2cm}
  \textbf{Comprehension Questions}\\
  How does pneumonia cause anemia?
  %\textbf{Open-Ended Questions}\\
  %What can cause anemia?\\\\
  %\textbf{More complex questions}\\
  %How does pneumonia cause anemia?
  %\begin{textblock*}{5cm}(9cm,1.8cm) 
  %\resizebox{0.7\textwidth}{!}{
  %\begin{tikzpicture}
    %\node[object, xshift=2cm,yshift=2cm, anchor=north east] (P1) {organ failure};
    %\node[object,xshift=2cm,yshift=2cm, anchor=north east, fill=tab20darkred!60,label={0: \huge $\mathbf{\downarrow}$}] (P2) {climate change};
    %\node[object, right= 0.5cm of P2] (P3) {ards};
    
   % \node (T1) {\textbf{Binary Causal Questions}};
   % \node[below = 0.3cm of T1] (T2) {Does pneumonia cause anemia?};

   \tikzset{person/.style={draw,rectangle,fill=tab20darkgreen!60,rounded corners,inner sep=5pt} }
  \tikzset{object/.style={draw,rectangle,fill=tab20darkblue!60,rounded corners, inner sep=5pt, minimum height=0.7cm}}
  \tikzset{type/.style={draw,rectangle,fill=lightgray,inner sep=5pt}}
  \tikzset{>=triangle 45}
  \tikzset{every picture/.style=thick}
  \tikzstyle{every node}=[font=\small]
    \tikzstyle{arrow} = [->, thick,>=stealth']
  
  
  \vspace{-0.5cm}
  \begin{textblock*}{5cm}(8cm,5.0cm) 
    \resizebox{0.7\textwidth}{!}{%
  \hspace{0.0cm}
  \begin{tikzpicture}
    \node[object, fill=white] (P1) {sepsis};
    \node[object, right= 0.5cm of P1] (P2) {anemia};
    \node[object, right= 0.5cm of P2, fill=white] (P3) {ards};
    
    \node[person, above=of P2] (T1) {pneumonia};

    \draw[arrow, line width=1.2pt] (T1) -- (P1);
    \draw[arrow, line width=1.2pt] (T1) -- (P2);
    \draw[arrow, line width=1.2pt] (T1) -- (P3);

  \end{tikzpicture}}
  \end{textblock*}
  %\end{tikzpicture}}
  %\end{textblock*}
  \end{columns}
\end{frame}

\begin{frame}{Causal Question Answering}
  \begin{itemize}
    \item<1-> Solution
    \begin{itemize}
      \item Perform \emph{walks on a knowledge graph}
      \item Formulated as a sequential decision problem\\via reinforcement learning
    \end{itemize}
    \item<2-> Advantages
    \begin{itemize}
      \item Answers are \emph{explainable}\\$\Rightarrow$ Can follow the reasoning chain
      \item Can answer \emph{binary and open-ended questions}
      %\item Performs better than walks learned\\via supervised learning
    \end{itemize}
    \item<3-> Contributions
    \begin{itemize}
      \item Approach to answer \emph{binary causal questions}\\via reinforcement learning
      \item Introduce an Actor-Critic (A2C) based agent with\\generalized advantage estimation (GAE)
      \item Supervised learning and reward shaping to deal with\\large action spaces and sparse rewards
    \end{itemize}
  \end{itemize}
   \tikzset{person/.style={draw,rectangle,fill=tab20darkgreen!60,rounded corners,inner sep=5pt, minimum height=0.7cm} }
  \tikzset{object/.style={draw,rectangle,fill=tab20darkgreen!60,rounded corners, inner sep=5pt, minimum height=0.7cm}}
  \tikzset{type/.style={draw,rectangle,fill=lightgray,inner sep=5pt}}
  \tikzset{>=triangle 45}
  \tikzset{every picture/.style=thick}
  \tikzstyle{every node}=[font=\small]
    \tikzstyle{arrow} = [->, thick,>=stealth']

  \begin{textblock*}{5cm}(9cm,1.8cm) 
  \resizebox{0.7\textwidth}{!}{
  \begin{tikzpicture}[remember picture]
    
    \node[object, xshift=2cm,yshift=2.2cm, anchor=north east] (P1) {fatigue};
    \node[object, right= 0.7cm of P1,fill=tab20darkblue!60] (P2) {sepsis};
    \node[object, right= 0.7cm of P2] (P3) {ards};
    
    \node[person, above=0.7cm of P2,fill=tab20darkblue!60] (T1) {pneumonia};
    \node[person, above=0.7cm of T1] (T2) {bacteria};
    \node[person, below=0.7cm of P3] (T3) {death};
    \node[person, below=0.7cm of T3] (P8) {grief};
    \node[person, below=0.7cm of P2,fill=tab20darkblue!60] (T4) {kidney failure};
    \node[person, below=0.7cm of T4,fill=tab20darkblue!60] (P7) {anemia};

    \node[above=0.7cm of T2] (T5) {\textbf{Does pneumonia cause anemia?}};

    \draw[arrow, line width=1.2pt] (T1) -- (P1);
    \draw[arrow, line width=1.2pt] (T1) -- (P2);
    \draw[arrow, line width=1.2pt] (T1) -- (P3);
    \draw[arrow, line width=1.2pt] (T2) -- (T1);
    \draw[arrow, line width=1.2pt] (P3) -- (T3);
    \draw[arrow, line width=1.2pt] (P2) -- (T4);
    \draw[arrow, line width=1.2pt] (T4) -- (P7);
    \draw[arrow, line width=1.2pt] (T3) -- (P8);

  \end{tikzpicture}}
  \end{textblock*}
\end{frame}


\begin{frame}{CauseNet}{Large-Scale Causal Knowledge Graph}
  \begin{itemize}
	  \item $\mathcal{K}=(\mathcal{E}, \mathcal{R})$: entities $\mathcal{E}$, relations $\mathcal{R}$
	  \item $\mathcal{R} = \{mayCause\}$:\\$\Rightarrow$ Claimed causal relations
	  %\item Extracted via linguistic pattern from\\web sources like Wikipedia and ClueWeb12
    \item Meta-information like the \emph{original sentence} \\and the \emph{URL} for each relation
	  \item Two configurations: \emph{CauseNet-Precision}\\and CauseNet-Full~{\scriptsize\cite{Heindorf2020Causenet}}
\end{itemize}

   \tikzset{person/.style={draw,rectangle,fill=tab20darkgreen!60,rounded corners,inner sep=5pt, minimum height=0.7cm} }
  \tikzset{object/.style={draw,rectangle,fill=tab20darkgreen!60,rounded corners, inner sep=5pt, minimum height=0.7cm}}
  \tikzset{type/.style={draw,rectangle,fill=lightgray,inner sep=5pt}}
  \tikzset{>=triangle 45}
  \tikzset{every picture/.style=thick}
  \tikzstyle{every node}=[font=\small]
    \tikzstyle{arrow} = [->, thick,>=stealth']
  
  

  \begin{textblock*}{5cm}(8.4cm,1.5cm) 
  \resizebox{0.8\textwidth}{!}{
  \begin{tikzpicture}[remember picture]
    \node[object, xshift=2cm,yshift=2cm, anchor=north east] (P1) {sepsis};
    \node[object, right= 0.5cm of P1] (P2) {anemia};
    \node[object, right= 0.5cm of P2] (P3) {ards};
    
    \node[person, above=0.7cm of P2] (T1) {pneumonia};
    \node[person, above=0.7cm of T1] (T2) {bacteria};
    \node[person, below=0.7cm of P3] (T3) {death};
    \node[person, below=0.7cm of P2] (T4) {fatigue};

    \draw[arrow, line width=1.2pt] (T1) -- (P1);
    \draw[arrow, line width=1.2pt] (T1) -- (P2);
    \draw[arrow, line width=1.2pt] (T1) -- (P3);
    \draw[arrow, line width=1.2pt] (T2) -- (T1);
    \draw[arrow, line width=1.2pt] (P3) -- (T3);
    \draw[arrow, line width=1.2pt] (P2) -- (T4);

  \end{tikzpicture}}
\end{textblock*}

	\begin{exampleblock}<2->{Causal Questions}
	\begin{itemize}
	  \item Does \underline{pneumonia} cause \underline{fatigue}?
	  \item Is \underline{sepsis} caused by \underline{pneumonia}?
	\end{itemize}
\end{exampleblock}

\end{frame}

\begin{frame}{Task and Problem Definition}
	\begin{block}{Causal Question Answering on Knowledge Graph $\mathcal{K}$}
    \begin{itemize}
	\item Input:
	\begin{itemize}
	  \item $\mathcal{K}=(\mathcal{E}, \mathcal{R})$ with $\mathcal{R} = \{cause\}$
    \item Question $q$ with exactly \emph{one cause} $e_c$ and \emph{one effect} $e_e$\\
    \textcolor{gray}{Does pneumonia ($= e_c$) cause anemia ($ = e_e$)?}
	\end{itemize}
	\item<2-> Output:
	\begin{itemize}
	    \item Path: $(e_c, e_1, \dots, e_e)$ with $e_c, e_i, e_e \in \mathcal{E}$
      \item If such a path can be found answer ``yes'' and ``no'' otherwise
    \end{itemize}
	\item<3-> Challenges:
	\begin{itemize}
	    \item Only \emph{one relation type} contrary to prior approaches\\$\Rightarrow$ Large action space
	    \item CauseNet-Precision has 80,223 entities
    \end{itemize}
  \end{itemize}
\end{block}
\end{frame}

\begin{frame}{Reinforcement Learning}
\begin{itemize}
  \item Policy gradient methods with policy network $\pi_{\theta} (a_t | s_t)$:
  \begin{equation*}
    \nabla_{\theta} J (\theta) = \mathbb{E}_{\pi_\theta} \left[ \sum_{t=0}^T \nabla_\theta \, \log(\pi_\theta (a_t | s_t)) \Psi_t \right] 
    \label{eq:polupdate}
  \end{equation*}
  \item<2-> Possible forms for $\Psi_t$~{\scriptsize\cite{Schulman2016GAE}}:
    \begin{equation*}
    R_t = \sum_{i=0}^{T-t} \gamma^{i} \, r_{t+i} \ \ \ \ \ \ \ \ \ \ \ \ \
      \mathcal{A}_t^\psi = R_t(\lambda) - V_\psi(s_t)
    \end{equation*}
  \item<3-> \emph{Monte-Carlo return} $R_t$ unbiased but high variance\\$\Rightarrow$ REINFORCE
  \item<3-> \emph{Advantage} $\mathcal{A}_t^{\psi}$ introduces value network $V_{\psi}(s_t)$ to \emph{reduce variance}\\$\Rightarrow$ Advantage Actor-Critic (A2C)~{\scriptsize\cite{Mnih2016A2C}}
  \item<3-> Using the $\lambda$-return $R_t(\lambda)$ in the advantage yields the generalized advantage estimation (GAE)~{\scriptsize\cite{Schulman2016GAE}}
\end{itemize}

\end{frame}

\begin{frame}{Environment}{Agent, States, Actions}
  \begin{columns}[T]
    \column{0.6\textwidth}
  \begin{itemize}
    \item Agent
    \begin{itemize}
      \item $\pi_\theta(a_t | s_t)$: policy network\\
       $\Rightarrow$ Distribution over actions
      \item $V_\psi(s_t)$: value network \\
      $\Rightarrow$ Reward from state $s_t$ onwards
    \end{itemize}
    \item<2-> States
    \begin{itemize}
      \item $s = (\mathbf{q}, e_t, \mathbf{e_t}, \mathbf{h_t}, e_e)$
      \item Question Embedding $\mathbf{q}$
      \item Current entity $e_t$, target entity $e_e$
      \item History $\mathbf{h_t}$ (LSTM hidden states)
    \end{itemize}
    \item<3-> Actions
    \begin{itemize}
      \item All neighboring entities of $e_t$:\\
      $A(s_t) = \{e | (e_t, r, e) \in \mathcal{K}\}$
    \end{itemize}
  \end{itemize}
  \column{0.4\textwidth}
  %\vspace{-1.0cm}
  \begin{figure}
    \centering
  \hspace*{-0.5cm}
  \includegraphics[clip, trim=7cm 19cm 5cm 2cm, width=1.3\textwidth]{rl.pdf}
  \end{figure}
   \tikzset{person/.style={draw,rectangle,fill=tab20darkgreen!60,rounded corners,inner sep=5pt} }
  \tikzset{object/.style={draw,rectangle,fill=tab20darkblue!60,rounded corners, inner sep=5pt, minimum height=0.7cm}}
  \tikzset{type/.style={draw,rectangle,fill=lightgray,inner sep=5pt}}
  \tikzset{>=triangle 45}
  \tikzset{every picture/.style=thick}
  \tikzstyle{every node}=[font=\small]
    \tikzstyle{arrow} = [->, thick,>=stealth']
  
  
  \vspace{-0.5cm}
    \resizebox{1.1\textwidth}{!}{%
  \hspace{0.0cm}
  \begin{tikzpicture}
    \node[object] (P1) {sepsis};
    \node[object, right= 0.5cm of P1, label={-90:$\mathbf{a_t = e_{t+1}}$}] (P2) {anemia};
    \node[object, right= 0.5cm of P2] (P3) {ards};
    \node[right= 0.1cm of P3] (P4) {\large $\biggl\} \, \, A(s_t)$};
    
    \node[person, above=of P2, label={90:$\mathbf{e_t}$}] (T1) {pneumonia};

    \draw[arrow, line width=1.2pt] (T1) -- (P1);
    \draw[arrow, line width=1.2pt] (T1) -- (P2);
    \draw[arrow, line width=1.2pt] (T1) -- (P3);

  \end{tikzpicture}}
  \end{columns}
\end{frame}

\begin{frame}{Environment}{Actions, Transitions, Rewards}
  \begin{columns}[T]
    \column{0.60\textwidth}
  \begin{itemize}
    \item Actions
    \begin{itemize}
      \item \textit{STAY} action: $A(s_t) = A(s_t) \cup \{STAY\}$
      \item Including \emph{inverse edges}
      %\item Embeddings: $\mathbf{a_t} = [\mathbf{s};\mathbf{a_t}]$ with\\original sentence $s$
    \end{itemize}
    \item<2-> Transitions
    \begin{itemize}
      \item $\delta(s_t, a_t) = s_{t+1}$ where $a_t = e_{t+1}$
      \item \emph{Deterministic} and entirely defined\\by the graph
    \end{itemize}
    \item<3-> Rewards
    \begin{itemize}
      \item $\mathcal{R}(s_{T-1}) = r_t$ with $r_t = 1$ if \emph{$e_{T-1} = e_e$} and $r_t = 0$ otherwise
      %$s_{T-1} = (\mathbf{q}, e_{T-1}, \mathbf{e_{T-1}}, \mathbf{h_{T-1}}, e_e)$ with 
      \item $0$ at all other time steps\\
      $\Rightarrow$ \emph{sparse reward}
    \end{itemize}
  \end{itemize}
  \column{0.4\textwidth}
  
  %\vspace{-0.2cm}
  \begin{figure}
    \centering
  \hspace*{-0.5cm}
  \includegraphics[clip, trim=7cm 19cm 5cm 2cm, width=1.3\textwidth]{rl.pdf}
  \end{figure}
   \tikzset{person/.style={draw,rectangle,fill=tab20darkgreen!60,rounded corners,inner sep=5pt} }
  \tikzset{object/.style={draw,rectangle,fill=tab20darkblue!60,rounded corners, inner sep=5pt, minimum height=0.7cm}}
  \tikzset{type/.style={draw,rectangle,fill=lightgray,inner sep=5pt}}
  \tikzset{>=triangle 45}
  \tikzset{every picture/.style=thick}
  \tikzstyle{every node}=[font=\small]
    \tikzstyle{arrow} = [->, thick,>=stealth']
  
  
  \vspace{-0.5cm}
    \resizebox{1.1\textwidth}{!}{%
  \hspace{0.0cm}
  \begin{tikzpicture}
    \node[object] (P1) {sepsis};
    \node[object, right= 0.5cm of P1, label={-90:$\mathbf{a_t = e_{t+1}}$}] (P2) {anemia};
    \node[object, right= 0.5cm of P2] (P3) {ards};
    \node[right= 0.1cm of P3] (P4) {\large $\biggl\} \, \, A(s_t)$};
    
    \node[person, above=of P2, label={90:$\mathbf{e_t}$}] (T1) {pneumonia};

    \draw[arrow, line width=1.2pt] (T1) -- (P1);
    \draw[arrow, line width=1.2pt] (T1) -- (P2);
    \draw[arrow, line width=1.2pt] (T1) -- (P3);

  \end{tikzpicture}}
  \end{columns}
\end{frame}

\begin{frame}{Agent}{Training}

\begin{itemize}
  \item Network Architecture
  \begin{itemize}
    \item LSTM to include the \emph{path history}
    \item Stack two feedforward networks on top\\$\Rightarrow$ For policy network and value network
  \end{itemize}

  \item<2-> Preprocessing of questions
  \begin{itemize}
    \item \emph{No negative information} in CauseNet\\$\Rightarrow$ Only use \emph{positive questions}
    \item Find $e_c$ and $e_e$ in graph via exact string matching
  \end{itemize}
  \item<3-> Sample rollouts
  \begin{itemize}
    \item Starting at $e_c$ sample \emph{rollouts of length $T$}
    \item If $e_e$ found, the agent should use the \textit{STAY} action
  \end{itemize}

  \begin{equation*}
    \uncover<3->{\text{rollout} = (} \uncover<4->{(s_0, a_0, r_0),} \uncover<5->{(s_1, a_1, r_1),} \uncover<6->{(s_2, a_2, r_2))}
  \end{equation*}
   \tikzset{person/.style={draw,rectangle,fill=tab20darkgreen!60,rounded corners,inner sep=5pt, minimum height=0.7cm} }
  \tikzset{object/.style={draw,rectangle,fill=tab20darkgreen!60,rounded corners, inner sep=5pt, minimum height=0.7cm}}
  \tikzset{type/.style={draw,rectangle,fill=lightgray,inner sep=5pt}}
  \tikzset{>=triangle 45}
  \tikzset{every picture/.style=thick}
  \tikzstyle{every node}=[font=\small]
    \tikzstyle{arrow} = [->, thick,>=stealth']

  \begin{textblock*}{5cm}(9.3cm,2.8cm) 
  \resizebox{0.65\textwidth}{!}{
  \begin{tikzpicture}[remember picture]
    %\node[object, xshift=2cm,yshift=2cm, anchor=north east] (P1) {organ failure};
    \uncover<5->{\node[object,xshift=2cm,yshift=2cm, anchor=north east, label={0:$\mathbf{s_2}, \mathbf{r_1 = 0}$}] (P2) {kidney failure};}
    %\node[object, right= 0.5cm of P2] (P3) {ards};
    
    \uncover<4->{\node[person, above=0.7cm of P2, label={0:$\mathbf{s_1}, \mathbf{r_0 = 0}$}] (T1) {sepsis};}
    \uncover<3->{\node[person, above=0.7cm of T1, label={0:$\mathbf{s_0}$}] (T2) {pneumonia};}
    %\node[person, below=0.7cm of P3] (T3) {death};
    \uncover<6->{\node[person, below=0.7cm of P2, label={0:$\mathbf{s_3}, \mathbf{r_2 = ?}$}] (T4) {anemia};}

    %\draw[->] (T1) -- (P1);
    \uncover<5->{\draw[arrow, line width=1.2pt] (T1) to node[right=0.04cm, pos=0.5]  {$\mathbf{a_1}$} (P2);}
    %\draw[->] (T1) -- (P3);
    \uncover<4->{\draw[arrow, line width=1.2pt] (T2) to node[right=0.04cm, pos=0.5]  {$\mathbf{a_0}$} (T1);}
    %\draw[->] (P3) -- (T3);
    \uncover<6->{\draw[arrow, line width=1.2pt] (P2) to node[right=0.04cm, pos=0.5]  {$\mathbf{a_2}$} (T4);}

  \end{tikzpicture}}
  \end{textblock*}
\end{itemize}


\end{frame}

\begin{frame}{Agent}{Update Rules}

\begin{itemize}
\item Using Synchronous Advantage Actor-Critic (A2C) with GAE
\item \emph{Actor:} policy network $\pi_{\theta}(a_t | s_t)$:
\begin{equation*}
  \nabla_{\theta} J(\theta) = - \frac{1}{B} \sum_{i}^{B} \sum_{t=0}^{T-2} \nabla_{\theta} \log(\pi_{\theta} (a_t | s_t)) \, \mathcal{A}_{t}^{\psi} + \beta H_{\pi_{\theta}}
\end{equation*}
  \begin{itemize}
  \item $B$: batch size, $T$: episode length, $\mathcal{A}_t^{\psi}$: GAE
  \item $H_{\pi_{\theta}}$: entropy regularization $\Rightarrow$ \emph{exploration vs. exploitation}
  \end{itemize}
\item<2-> \emph{Critic:} value network $V_{\psi}(s_t)$:
  \begin{equation*}
    \nabla_{\psi} J(\psi)= \frac{1}{B (T-1)} \sum_{i}^{B} \sum_{t=0}^{T-2} \nabla_{\psi} (R_t(\lambda) - V_{\psi}(s_t))^2
  \end{equation*}
  \begin{itemize}
  \item MSE between $\lambda$-return and value network predictions $V_{\psi}(s_t)$
  \end{itemize}
\end{itemize}
\end{frame}

\begin{frame}{Agent}{Inference}

\begin{block}{Inference Time}
  \begin{itemize}
  \item Receive positive and negative questions
  \item Answer ``yes'' if a path was found and ``no'' otherwise
  \end{itemize}
\end{block}

\vspace{-0.2cm}
\begin{itemize}
  \item<2-> Greedy decoding:
  \begin{itemize}
    \item Choose action: $arg\,max_{a_t \in \mathcal{A}(s_t)} \ \pi_{\theta}(a_t | s_t)$
    \item \emph{Myopic} behavior
  \end{itemize}
  \item<3-> Beam search:
  \begin{itemize}
    \item Always keep set of best partial solutions (paths)
    \item Paths \emph{ranked by their probability}
    \item Probability of path $p = (e_c,e_1, \dots, e_{T-1})$ with $e_t = a_{t-1}$:
    \begin{equation*}
  \mathbb{P}(p) = \prod_{t=0}^{T-2} \pi_{\theta}(a_t |s_t)
      \end{equation*}
  \end{itemize}
   \tikzset{person/.style={draw,rectangle,fill=tab20darkgreen!60,rounded corners,inner sep=5pt} }
  \tikzset{object/.style={draw,rectangle,fill=tab20darkblue!60,rounded corners, inner sep=5pt, minimum height=0.7cm}}
  \tikzset{type/.style={draw,rectangle,fill=lightgray,inner sep=5pt}}
  \tikzset{>=triangle 45}
  \tikzset{every picture/.style=thick}
  \tikzstyle{every node}=[font=\small]
    \tikzstyle{arrow} = [->, thick,>=stealth']
  
  
  \vspace{-0.5cm}
  \begin{textblock*}{5cm}(9cm,4.0cm) 
    \resizebox{0.65\textwidth}{!}{%
  \hspace{0.0cm}
  \begin{tikzpicture}
    \uncover<2->{\node[object, fill=white] (P1) {sepsis};}
    \uncover<2->{\node[object, right= 0.5cm of P1] (P2) {anemia};}
    \uncover<2->{\node[object, right= 0.5cm of P2, fill=white] (P3) {ards};}
    
    \uncover<2->{\node[person, above=of P2] (T1) {pneumonia};}

    \uncover<2->{\draw[arrow, line width=1.2pt] (T1) to node[left=0.04cm, pos=0.5]  {$0.2$} (P1);}
    \uncover<2->{\draw[arrow, line width=1.2pt] (T1) to node[right=0.00cm, pos=0.5]  {$\mathbf{0.6}$} (P2);}
    \uncover<2->{\draw[arrow, line width=1.2pt] (T1) to node[right=0.04cm, pos=0.5]  {$0.2$} (P3);}

  \end{tikzpicture}}
  \end{textblock*}
\end{itemize}

\end{frame}

\begin{frame}{Walkthrough}{Beam Width = 2}
\begin{figure}
    \centering
    \small
    \begin{tikzpicture}
        %\tikzstyle{vertex}=[draw, rectangle, rounded corners, minimum width=1.8cm, minimum height=1.4cm, line width=0.8pt]
        \tikzstyle{vertex}=[draw, rounded corners, line width=0.8pt, inner sep=4pt, minimum height = 0.7cm]
        \tikzstyle{box}=[minimum height=0.8cm, minimum width = 6cm, inner sep=4pt, rounded corners]
        \tikzstyle{arrow} = [->, thick,>=stealth']
        
        
        \node[vertex, , fill=tab20darkgreen!60] (N1) {pneumonia};
        %\node[vertex, left = 3cm of MCQ] (N1) {pneumonia};
        \node[vertex, left = 1.3cm of N1, fill=tab20darkblue!60] (bacteria) {bacteria};
        \node[vertex, above left = 1cm and 1cm of N1] (inflammation) {inflammation};
        \node[vertex, right = 1.5cm of N1] (fevers) {fevers};
        \node[vertex, above = 1cm  of N1, fill=tab20darkblue!60] (sepsis) {sepsis};
        \node[vertex, below = 1cm of N1, fill=tab20darkblue!60] (ards) {ARDS};
        \node[vertex, right = 2.05cm of ards, fill=tab20darkblue!60] (death) {death};
        
        \node[vertex, right = 1cm  of sepsis, fill=tab20darkblue!60] (hypoglycemia) {hypoglycemia};
        
         \node[draw, rounded corners, line width=0.5pt, inner sep=4pt,below left = 0.75cm and -1.5cm of death, fill=brass!60, minimum height=2cm, minimum width = 2.4cm, align=center] (LM) {Language \\ Model};
         
         
         
         \node[left color=white, right color=brass!40, above left = -1.15cm and 0cm of LM, fill=brass!40, minimum height=1.1cm, minimum width = 4cm, align=center] (q1) {How does pneumonia ..};
         \node[left color=white, right color=brass!40, below = 0.1cm of q1, fill=brass!40, minimum height=0.73cm, minimum width = 4cm, align=center] (q2) {Path (P3)};
         \node[left = 0.2cm of q1, fill=white, minimum height=1.1cm, minimum width = 1.5cm, align=center] (t1) {\textbf{Question:}};
         \node[left = 0.2cm of q2, fill=white, minimum height=1.1cm, minimum width = 1.5cm, align=center] (t2) {\textbf{Context:}};
         \node[right color=white, left color=brass!40, right = 0cm of LM, fill=brass!40, minimum height=1.4cm, minimum width = 4cm, align=center] (q3) {Pneumonia can induce ARDS which ..};

        \node[box, fill=brass!40, label=above:{\textbf{(1) Binary Question}}, above right = 1cm and 3.4cm of N1] (MCQ)  {Does pneumonia cause hypoglycemia?};
        \node[box, below = of MCQ, fill=brass!40, label=above:{\textbf{(2) Cause/Effect Question}}] (CQ)  {What causes pneumonia?};
        \node[box, below = of CQ, fill=brass!40, label=above:{\textbf{(3) Complex Question}}] (C)  {How does pneumonia cause death?};
        
        \node[above left = 0.3cm and -0.4cm of inflammation] (a) {\textbf{a)}};
        \node[above left = 0.2cm and -0.5cm of MCQ] (b) {\textbf{b)}};
        \node[below left = 0.35cm and 3.4cm of ards] (c) {\textbf{c)}};
        
        \draw[arrow, line width=1.2pt] (bacteria) to  node[below=0.1cm, pos=0.5]  {\textbf{(P2)}} (N1);
        \draw[arrow, line width=1.2pt] (N1) to  node[right=0.1cm, pos=0.5]  {\textbf{(P1)}} (sepsis);
        \draw[arrow, line width=1.2pt] (N1) to  node[right=0.1cm, pos=0.5]  {\textbf{(P3)}} (ards);
        \draw[arrow] (N1) -- (fevers);
        \draw[arrow] (N1) -- (inflammation);
        
        \draw[arrow, line width=1.2pt] (sepsis) to  node[above=0.1cm, pos=0.5]  {\textbf{(P1)}} (hypoglycemia);
        \draw[arrow, line width=1.2pt] (ards) to  node[below=0.1cm, pos=0.5]  {\textbf{(P3)}} (death);

        \draw[thick,rounded corners]     ($(inflammation.north west)+(-0.2,0.89)$) rectangle ($(death.south east)+(0.5,-0.3)$);
        \draw[thick,rounded corners]     ($(MCQ.north west)+(-0.1,0.8)$) rectangle ($(C.south east)+(0.5,-0.13)$);
        
        \draw[thick,rounded corners]     ($(ards.south west)+(-4,-0.3)$) rectangle ($(C.south east)+(0.5,-3)$);
    \end{tikzpicture}
    \caption{A snapshot from CauseNet \textbf{a)} together with concrete examples for each question type \textbf{b)}. The annotated paths in the graph correspond to possible answers for each question, respectively. Note how question 3) asks for an explanation of a causal relation. In this case, that pneumonia causes death by inducing acute respiratory distress syndrome (ARDS). For this question type, we take the path(s) found by the agent and feed them into a language model as additional context \textbf{c)}.}
    \label{fig:graph_example}
\end{figure}
\end{frame}

\begin{frame}{Bootstrapping with Supervised Learning}

\begin{itemize}
  \item Problem: large action space + sparse rewards
  \begin{itemize}
    \item \emph{Slow convergence}
    \item Guidance in the beginning
  \end{itemize}
  \item<2-> Generating expert demonstrations~{\scriptsize\cite{Xiong2017DeePpath}}:
  \begin{itemize}
    \item Paths from a \emph{breadth-first search (BFS)}
    \item Preprocessing step for each question $q$\\\textcolor{gray}{Find path between $e_c$ and $e_e$}
  \end{itemize}
  \item<3-> Supervised gradient update:
  \begin{itemize}
    \item \emph{$r_t = 1$ at each time step}
    \item Batch size $B$, episode length $T$
    \item Entropy regularization $H_{\pi_{\theta}}$
  \end{itemize}

    \begin{equation*}
      \nabla_{\theta} J(\theta) = - \frac{1}{B} \sum_{i}^{B} \sum_{t=0}^{T-2} \nabla_{\theta} \log(\pi_{\theta} (a_t | s_t)) \, r_t + \beta H_{\pi_\theta}
    \end{equation*}

   \tikzset{person/.style={draw,rectangle,fill=tab20darkgreen!60,rounded corners,inner sep=5pt, minimum height=0.7cm} }
  \tikzset{object/.style={draw,rectangle,fill=tab20darkgreen!60,rounded corners, inner sep=5pt, minimum height=0.7cm}}
  \tikzset{type/.style={draw,rectangle,fill=lightgray,inner sep=5pt}}
  \tikzset{>=triangle 45}
  \tikzset{every picture/.style=thick}
  \tikzstyle{every node}=[font=\small]
    \tikzstyle{arrow} = [->, thick,>=stealth']

  \begin{textblock*}{5cm}(9cm,1.8cm) 
  \resizebox{0.65\textwidth}{!}{
  \begin{tikzpicture}[remember picture]
    %\node[object, xshift=2cm,yshift=2cm, anchor=north east] (P1) {organ failure};
    \uncover<2->{\node[object,xshift=2cm,yshift=2cm, anchor=north east, label={0:$\mathbf{s_2}, \mathbf{r_1 = 1}$}] (P2) {kidney failure};}
    %\node[object, right= 0.5cm of P2] (P3) {ards};
    
    \uncover<2->{\node[person, above=0.7cm of P2, label={0:$\mathbf{s_1}, \mathbf{r_0 = 1}$}] (T1) {sepsis};}
    \uncover<2->{\node[person, above=0.7cm of T1, label={0:$\mathbf{s_0}$}] (T2) {pneumonia};}
    %\node[person, below=0.7cm of P3] (T3) {death};
    \uncover<2->{\node[person, below=0.7cm of P2, label={0:$\mathbf{s_3}, \mathbf{r_2 = 1}$}] (T4) {anemia};}

    %\draw[->] (T1) -- (P1);
    \uncover<2->{\draw[arrow, line width=1.2pt] (T1) to node[right=0.04cm, pos=0.5]  {$\mathbf{a_1}$} (P2);}
    %\draw[->] (T1) -- (P3);
    \uncover<2->{\draw[arrow, line width=1.2pt] (T2) to node[right=0.04cm, pos=0.5]  {$\mathbf{a_0}$} (T1);}
    %\draw[->] (P3) -- (T3);
    \uncover<2->{\draw[arrow, line width=1.2pt] (P2) to node[right=0.04cm, pos=0.5]  {$\mathbf{a_2}$} (T4);}

  \end{tikzpicture}}
  \end{textblock*}
\end{itemize}

\end{frame}


\begin{frame}{Reward Shaping}

\begin{itemize}
    \item Problem: sparse rewards
    \begin{itemize}
      \item Agent only receives a reward if $e_{T-1} = e_e$
      \item Rarely finds correct paths at the start
    \end{itemize}
    \item<2-> Introduce \emph{auxiliary reward}:
    \begin{itemize}
      \item Score \emph{last node} according to its \emph{relevance}\\to the question~{\scriptsize\cite{Yasunaga2021QAGNN}}
    \end{itemize}
    \item<3-> Updated reward:
    \begin{equation*}
      \mathcal{R}'(s_{T-1}) =
        \begin{cases}
          \mathcal{R}(s_{T-1}), & \mathcal{R}(s_{T-1}) = 1 \\
          Score(q, e_{T-1}) \cdot \omega, & \text{otherwise}
        \end{cases}
    \end{equation*}
    \item<3-> Where $Score$ is defined as:
    \begin{equation*}
      Score(q, e_{T-1}) = LM_{head}(LM_{enc}([q;e_{T-1}]))
    \end{equation*}
    \vspace{-0.5cm}
    \begin{itemize}
    \item $LM_{head}$: feedforward network, $LM_{enc}$: language model encoder
    \end{itemize}
\end{itemize}

   \tikzset{person/.style={draw,rectangle,fill=tab20darkgreen!60,rounded corners,inner sep=5pt, minimum height=0.7cm} }
  \tikzset{object/.style={draw,rectangle,fill=tab20darkgreen!60,rounded corners, inner sep=5pt, minimum height=0.7cm}}
  \tikzset{type/.style={draw,rectangle,fill=lightgray,inner sep=5pt}}
  \tikzset{>=triangle 45}
  \tikzset{every picture/.style=thick}
  \tikzstyle{every node}=[font=\small]
  \tikzstyle{arrow} = [->, thick,>=stealth']
  \tikzstyle{box}=[minimum height=0.8cm, minimum width = 6cm, inner sep=4pt, rounded corners]

  \begin{textblock*}{5cm}(9cm,1.8cm) 
  \resizebox{0.7\textwidth}{!}{
  \begin{tikzpicture}[remember picture]
    %\node[object, xshift=2cm,yshift=2cm, anchor=north east] (P1) {organ failure};
    \uncover<2->{\node[object,xshift=2cm,yshift=2cm, anchor=north east, fill=tab20darkred!60,label={0: \huge $\mathbf{\downarrow}$}] (P2) {climate change};}
    %\node[object, right= 0.5cm of P2] (P3) {ards};
    
    \uncover<2->{\node[person, above=0.7cm of P2, label={0: \huge $\mathbf{\uparrow}$}] (T1) {sepsis};}
    \uncover<2->{\node[above=0.7cm of T1] (T2) {\textbf{Does pneumonia cause anemia?}};}

  \end{tikzpicture}}
  \end{textblock*}
\end{frame}

\begin{frame}{Evaluation}



\begin{center}
\footnotesize
  \renewcommand{\arraystretch}{0.95}
  \renewcommand{\tabcolsep}{6.4pt}
  \begin{tabular}{lccccr} 
    \toprule
    \multicolumn{6}{c}{\textbf{MS MARCO}}\\
    \midrule
    & \textbf{Accuracy} & \textbf{$\mathbf{F_1}$} & \textbf{Recall} & \textbf{Precision} & \textbf{|Nodes|}\\
    \midrule
    %Agent 1-Hop & 0.36 & & \\
    Agent 2-Hop & 0.460 & 0.562 & 0.408 & \textbf{0.901} & \textbf{26}\\
    Agent 3-Hop & 0.529 & 0.648 & 0.511 & 0.884 & 27 \\
    \midrule
    %BFS 1-Hop & 0.259 & 0.241 & 0.139 & \textbf{0.912} & 57 \\ 
    BFS 2-Hop & 0.494 & 0.612 & 0.471 & 0.875 & 1,727 \\ 
    BFS 3-Hop & 0.589 & 0.714 & 0.605 & 0.871 & 3,339\\ 
    \midrule
    UnifiedQA-v2 & \textbf{0.722} & \textbf{0.828} & \textbf{0.789} & 0.871 & -- \\
    %UnifiedQA-v2 | CauseNet 3-Hop & 0.787 & 0.877 & 0.897 & 0.858 & --  \\
    %UnifiedQA-v2 | Agent 3-Hop & 0.779 & 0.872 & 0.883 & 0.860 & -- \\
    %UnifiedQA-v2 --- Context & 0.661 & 0.789 & 0.740 & 0.842 & --\\
    %\midrule
    %Majority Baseline (True) & 0.848  & 0.9177 & 1.000 & 0.848 & --\\
    \bottomrule
\end{tabular}
%\end{table}
%\begin{table}[t]
\centering
\renewcommand{\tabcolsep}{6.47pt}
\hspace{-0.20cm}
%\caption{Evaluation SemEval.}
\begin{tabular}{lccccr} 
    \toprule
    \multicolumn{6}{c}{\textbf{SemEval}}\\
    \midrule
    & \textbf{Accuracy} & \textbf{$\mathbf{F_1}$} & \textbf{Recall} & \textbf{Precision} & \textbf{|Nodes|}\\
    \midrule
    %Agent 1-Hop & 0.36 & & \\
    Agent 2-Hop &  0.769 & 0.714 & 0.575 & \textbf{0.943} & \textbf{27}\\
    Agent 3-Hop & 0.775 & 0.727 & 0.598 & 0.929 & 29\\
    \midrule
    %BFS 1-Hop & 0.665 & 0.508 & 0.345 & \textbf{0.968} & 35 \\ 
    BFS 2-Hop & \textbf{0.815} & \textbf{0.787} & 0.678 & 0.937 & 1,565 \\ 
    BFS 3-Hop & 0.751 & 0.754 & 0.759 & 0.750 & 3,687\\ 
    \midrule
    UnifiedQA-v2 & 0.497 & 0.653 & \textbf{0.943} & 0.500 & --\\
    %UnifiedQA-v2 | CauseNet 3-Hop & 0.520 & 0.677 & 1.000 & 0.512 & -- \\
    %UnifiedQA-v2 | Agent 3-Hop & 0.520 & 0.675 & 0.989 & 0.512 & -- \\
    %UnifiedQA-v2 --- Context & 0.566 & 0.651 & 0.805 & 0.547 & -- \\
    %\midrule
    %Majority Baseline (True) & 0.503 & 0.669 & 1.000 & 0.503 & -- \\
    \bottomrule
\end{tabular}
\end{center}



\end{frame}


\begin{frame}{Ablation Study}

\begin{center}
  \setlength{\tabcolsep}{2pt}
  \renewcommand{\arraystretch}{1.08}
  \footnotesize
  \begin{tabular}{lcccccccc} 
		\toprule
		& \multicolumn{4}{c}{\textbf{MS MARCO}} & \multicolumn{4}{c}{\textbf{SemEval}}  \\
		\cmidrule(lr{.6em}){2-5} \cmidrule(l{0.3em}){6-9}
		&\textbf{A} & \textbf{$\mathbf{F_1}$} & \textbf{R} & \textbf{P} & \textbf{A} & \textbf{$\mathbf{F_1}$} & \textbf{R} & \textbf{P}\\
		\midrule
		Agent 2-Hop & \textbf{0.460} & \textbf{0.562} & \textbf{0.408} & 0.901 & \textbf{0.769} & \textbf{0.714} & \textbf{0.575} & 0.943 \\ 
		\midrule
		$\mathbf{-}$ Beam Search & 0.293 & 0.306 &0.184 & \textbf{0.911} & 0.613 & 0.374 &0.230& \textbf{1.000} \\
		$\mathbf{-}$ Supervised Learning & 0.342 & 0.397 & 0.257 & 0.891 & 0.682 & 0.538 & 0.369 & \textbf{1.000} \\
		$\mathbf{-}$ Actor-Critic & 0.441 & 0.539 & 0.386 & 0.896 & 0.740 & 0.657 &0.494& 0.977 \\
		$\mathbf{-}$ Inverse Edges & 0.422 & 0.513 &0.359& 0.899 & 0.740 & 0.651 & 0.483 & \textbf{1.000} \\
		$\mathbf{+}$ Reward Shaping (0.1) & 0.449 & 0.548 & 0.395 & 0.898 & 0.757 & 0.691 & 0.540 & 0.959 \\
		$\mathbf{+}$ Reward Shaping (1.0) & 0.403 &0.489 & 0.336 & 0.893 & \textbf{0.769} & 0.706 & 0.552 & 0.980 \\
		\bottomrule
	\end{tabular}
\end{center}

%\begin{itemize}
  %\item Beam search most important
  %\item Beam search most important
%\end{itemize}

\end{frame}

\begin{frame}{Effects of Supervised Learning}{Accuracy}
\begin{figure}
	\centering

    \resizebox{0.9\textwidth}{!}{%
	\begin{tikzpicture}
		\begin{axis}
			[	%legend pos=outer north east,
				legend cell align={left},
				legend style={legend pos=south east},
				xlabel=steps,
				ylabel=accuracy,
				grid=major,
				xtick={0, 20, 40, 60, 80, 100, 120, 140, 160, 180, 200},
				xticklabels={0, 200 , 400, 600, 800, 1000, 1200, 1400, 1800, 1900, 2000},
				ytick={0.66, 0.68, 0.70, 0.72, 0.74, 0.76, 0.78, 0.80},
				yticklabels={0.66, 0.68, 0.70, 0.72, 0.74, 0.76, 0.78, 0.80},
				xmin=0,
				xmax=200,
				ymin=0.66,
				ymax=0.80,
				width=14cm,
				height=9cm]
			%\addplot[mark=none, tab20darkgreen, very thick] table [x=Step, y=r_semeval_400_1.0_32_5000_ff - accuracy, col sep=comma] {data/supervised_accuracy.csv};
			%\addlegendentry{400 Steps}
			\addplot[mark=none, tab20darkred, very thick] table [x=Step, y=r_semeval_300_1.0_32_5000_ff - accuracy, col sep=comma] {data/supervised_accuracy.csv};
			\addlegendentry{300 Steps}
			\addplot[mark=none, tab20darkblue, very thick] table [x=Step, y=r_semeval_200_1.0_32_5000_ff - accuracy, col sep=comma] {data/supervised_accuracy.csv};
			\addlegendentry{200 Steps}
			\addplot[mark=none, tab20darkorange, very thick] table [x=Step, y=r_semeval_100_1.0_32_5000_ff - accuracy, col sep=comma] {data/supervised_accuracy.csv};
			\addlegendentry{100 Steps}
		\end{axis}
	\end{tikzpicture}}
\end{figure}
\end{frame}

\begin{frame}{Effects of Supervised Learning}{Number of Explored Paths + Entropy of Policy Network}
\begin{figure}[ht]
	\begin{minipage}{.45\linewidth}
	  \centering
	\resizebox{1.26\textwidth}{!}{%
\begin{tikzpicture}
	\hspace{-1.5cm}
	\begin{axis}
		[	%legend pos=outer north east,
			legend cell align={left},
			legend style={legend pos=north west},
			xlabel=steps,
			ylabel=|paths|,
			%ylabel shift = 0.1 pt,
			scaled y ticks = false,
			grid=major,
			xtick={0, 50, 100, 150, 200},
			xticklabels={0, 500, 1000, 1500, 2000},
			ytick={0, 20000, 40000, 60000, 80000},
			yticklabels={0, 20000, 40000, 60000, 80000},
			xmin=0,
			xmax=200,
			ymin=0,
			ymax=80000,
			width=8cm,
			height=8cm]
		%\addplot[mark=none, tab20darkgreen, very thick] table [x=Step, y=r_semeval_400_1.0_32_5000_ff - unique_paths, col sep=comma] {data/supervised_paths.csv};
		%\addlegendentry{400 Steps}
		\addplot[mark=none, tab20darkred, very thick] table [x=Step, y=r_semeval_300_1.0_32_5000_ff - unique_paths, col sep=comma] {data/supervised_paths.csv};
		\addlegendentry{300 Steps}
		\addplot [mark=none, tab20darkblue, very thick]table [x=Step, y=r_semeval_200_1.0_32_5000_ff - unique_paths, col sep=comma] {data/supervised_paths.csv};
		\addlegendentry{200 Steps}
		\addplot [mark=none, tab20darkorange, very thick]table [x=Step, y=r_semeval_100_1.0_32_5000_ff - unique_paths, col sep=comma] {data/supervised_paths.csv};
		\addlegendentry{100 Steps}
		\addplot [mark=none, tab20darkgreen, very thick]table [x=Step, y=r_semeval_no_super_5000_ff - unique_paths, col sep=comma] {data/supervised_paths.csv};
		\addlegendentry{0 Steps}
	\end{axis}
\end{tikzpicture}}
	\end{minipage}
	%\hspace*{0.7cm}
	\begin{minipage}{.45\linewidth}
	  \centering
	\resizebox{1.2\textwidth}{!}{%
\begin{tikzpicture}
	\begin{axis}
		[	%legend pos=outer north east,
			legend cell align={left},
			legend style={legend pos=north east},
			xlabel=steps,
			ylabel=entropy,
			grid=major,
			xtick={0, 50, 100, 150, 200},
			xticklabels={0, 500, 1000, 1500, 2000},
			ytick={0.0, 0.5, 1.0, 1.5, 2.0, 2.5, 3.0, 3.5},
			yticklabels={0.0, 0.5, 1.0, 1.5, 2.0, 2.5, 3.0, 3.5},
			xmin=0,
			xmax=200,
			ymin=0.0,
			ymax=3.5,
			width=8cm,
			height=8cm]
		%\addplot[mark=none, tab20darkgreen, very thick] table [x=Step, y=r_semeval_400_1.0_32_5000_ff - entropy, col sep=comma] {data/supervised_entropy.csv};
		%\addlegendentry{400 Steps}
		\addplot[mark=none, tab20darkred, very thick] table [x=Step, y=r_semeval_300_1.0_32_5000_ff - entropy, col sep=comma] {data/supervised_entropy.csv};
		\addlegendentry{300 Steps}
		\addplot [mark=none, tab20darkblue, very thick]table [x=Step, y=r_semeval_200_1.0_32_5000_ff - entropy, col sep=comma] {data/supervised_entropy.csv};
		\addlegendentry{200 Steps}
		\addplot [mark=none, tab20darkorange, very thick]table [x=Step, y=r_semeval_100_1.0_32_5000_ff - entropy, col sep=comma] {data/supervised_entropy.csv};
		\addlegendentry{100 Steps}
		\addplot[mark=none, tab20darkgreen, very thick] table [x=Step, y=r_semeval_no_super_5000_ff - entropy, col sep=comma] {data/supervised_entropy.csv};
		\addlegendentry{0 Steps}
	\end{axis}
\end{tikzpicture}}
	\end{minipage}
  \end{figure}
\end{frame}

\begin{frame}{Conclusion}
\begin{itemize}
    \item Summary
    \begin{itemize}
        \item Introduced an Actor-Critic (A2C) based agent\\to answer causal questions
        \item Extended with supervised learning and reward shaping
    \end{itemize}
    \item<2-> Conclusion
    \begin{itemize}
        \item Supervised learning provides an effective foundation
        \item Effectively \emph{prunes the search space by around 99\%}
        \item Paths can be used as \emph{explanations}\\$\Rightarrow$ Use meta-information to \emph{verify the claims}
    \end{itemize}
    \item<3-> Future Work
    \begin{itemize}
        \item Extend to \emph{open-ended questions}\\$\Rightarrow$Straightforward extension via majority voting
        \item Add different causal knowledge graphs
        \item Explore ways to consider negative\\causal questions during training
    \end{itemize}

   \tikzset{person/.style={draw,rectangle,fill=tab20darkgreen!60,rounded corners,inner sep=5pt, minimum height=0.7cm} }
  \tikzset{object/.style={draw,rectangle,fill=tab20darkgreen!60,rounded corners, inner sep=5pt, minimum height=0.7cm}}
  \tikzset{type/.style={draw,rectangle,fill=lightgray,inner sep=5pt}}
  \tikzset{>=triangle 45}
  \tikzset{every picture/.style=thick}
  \tikzstyle{every node}=[font=\small]
    \tikzstyle{arrow} = [->, thick,>=stealth']

  \begin{textblock*}{5cm}(9cm,3.8cm) 
  \resizebox{0.7\textwidth}{!}{
  \begin{tikzpicture}[remember picture]
    
    \node[object, xshift=2cm,yshift=2.2cm, anchor=north east] (P1) {fatigue};
    \node[object, right= 0.7cm of P1] (P2) {sepsis};
    \node[object, right= 0.7cm of P2] (P3) {ards};
    
    \node[person, above=0.7cm of P2] (T1) {pneumonia};
    \node[person, above=0.7cm of T1] (T2) {bacteria};
    \node[person, below=0.7cm of P3] (T3) {death};
    \node[person, below=0.7cm of T3] (P8) {grief};
    \node[person, below=0.7cm of P2] (T4) {kidney failure};
    \node[person, below=0.7cm of T4] (P7) {anemia};

    %\node[above=0.7cm of T2] (T5) {\textbf{Does pneumonia cause anemia?}};

    \draw[arrow, line width=1.2pt] (T1) -- (P1);
    \draw[arrow, line width=1.2pt] (T1) -- (P2);
    \draw[arrow, line width=1.2pt] (T1) -- (P3);
    \draw[arrow, line width=1.2pt] (T2) -- (T1);
    \draw[arrow, line width=1.2pt] (P3) -- (T3);
    \draw[arrow, line width=1.2pt] (P2) -- (T4);
    \draw[arrow, line width=1.2pt] (T4) -- (P7);
    \draw[arrow, line width=1.2pt] (T3) -- (P8);

  \end{tikzpicture}}
  \end{textblock*}
\end{itemize}
\end{frame}

\begin{frame}[allowframebreaks]{References}{~}
  \scriptsize
  \bibliographystyle{apalike}
  %\renewcommand{\bibfont}{\fontsize{8pt}{9pt}\selectfont}
  \bibliography{ref}
  
  \end{frame}

%\begin{frame}{Thank you!}
%\centering
%	\vspace{2.5cm}
%	{\huge\textbf{Questions?}}
%	\\\vspace{2.5cm}
%	{\large Data Science Group at Paderborn University}
%	\\\vspace{.1cm}
%	\centering
%	\begin{tabular}{rl}
%	Web: & \href{https://dice-research.org/}{dice-research.org} \\
%	Code: & \href{https://github.com/dice-group}{github.com/dice-group} \\
%	Twitter: & \href{https://twitter.com/DiceResearch}{@DiceResearch}
%	\end{tabular}
%\end{frame}

\begin{frame}{$\mathbf{\lambda}$-return \& GAE}
  \begin{itemize}
    \item N-step returns:
    \begin{equation*}
      R_t^{(n)} = \sum_{i=0}^{n-1} \gamma^i \, r_{t+i} + \gamma^n V_{\psi} (s_{t+n}) 
    \end{equation*}
    \item Which $n$ is best?
    \item $\lambda$-return exponentially-weighted average of $n$-step returns:
    \begin{equation*}
      %\vspace{-1cm}
      R_t(\lambda) = (1- \lambda) \sum_{n=1}^{\infty} \lambda^{n-1} R_t^{(n)} = (1-\lambda) \sum_{n=1}^{T-t-1} \lambda^{n-1} \, R_t^{(n)} + \lambda^{T-t-1} R_t^{(T-t)} 
    \end{equation*}
  \end{itemize}
\end{frame}

\begin{frame}{Network Architecture}
\begin{itemize}
    \item LSTM:
    \begin{equation*}
      \mathbf{h_t}=
        \begin{cases}
          LSTM(\mathbf{0}; [\mathbf{q}, \mathbf{e_c}]), & \text{if}\ t=0 \\
          LSTM(\mathbf{h_{t-1}}, [\mathbf{q}; \mathbf{e_t}]), & \text{otherwise}
        \end{cases}
    \end{equation*}
    \item Policy network $\pi_{\theta}(a_t | s_t)$:
    \begin{equation*}
      \begin{split}
      \pi_\theta(a_t | s_t) = \sigma(\mathbf{A_t} \times W_2 \times ReLU(W_1 \times \mathbf{h_t})) \\
      a_t \sim Categorical(\pi_\theta(a_t | s_t))
      \end{split}
    \end{equation*}
    \begin{itemize}
      \item $\mathbf{A_t} \in \mathcal{R}^{|A(s_t)| \times 2d}$ embeddings of actions $a_t \in \mathcal{A}(s_t)$
      \item $\sigma$: softmax operator
    \end{itemize}
    \item Value network $V_{\psi}(s_t)$:
    \begin{equation*}
      V_{\psi}(s_t) = W_4 \times ReLU(W_3 \times \mathbf{h_t})
    \end{equation*}

\end{itemize}
\end{frame}

\begin{frame}{Entropy}
  \begin{equation*}
    H_{\pi_\theta} = \frac{1}{B(T-1)} \sum_{i}^{B} \sum_{t=0}^{T-2} (- \hspace{-0.3cm}\sum_{a_t \in \mathcal{A}(s_t)} \pi_\theta(a_t | s_t)  \log \pi_\theta(a_t | s_t))
  \end{equation*}
\end{frame}

\begin{frame}{Beam Search Evaluation}
  \begin{figure}
    \centering
	\resizebox{1.0\textwidth}{!}{%
    \begin{tikzpicture}
      \begin{axis}
        [	%legend pos=outer north east,
          legend cell align={left},
          legend style={legend pos=south east},
          xlabel=steps,
          ylabel=accuracy,
          grid=major,
          xtick={0, 40, 80, 120, 160, 200, 240, 280, 320, 360, 400},
          xticklabels={0, 400, 800, 1200, 1600, 2000, 2400, 2800, 3200, 3600, 4000},
          ytick={0.1, 0.15, 0.2, 0.25, 0.3, 0.35, 0.4},
          yticklabels={0.1, 0.15, 0.2, 0.25, 0.3, 0.35, 0.4},
          xmin=0,
          xmax=400,
          ymin=0.1,
          ymax=0.4,
          width=14cm,
          height=8cm]
        \addplot [mark=none, tab20darkgreen, very thick]table [x=Step, y=r_msmarco_bm_50_5000_ff - accuracy, col sep=comma] {data/beam_path_inverse.csv};
        \addlegendentry{Width: 50}
        \addplot [mark=none, tab20darkorange, very thick]table [x=Step, y=r_msmarco_bm_10_5000_ff - accuracy, col sep=comma] {data/beam_path_inverse.csv};
        \addlegendentry{Width: 10}
        \addplot [mark=none, tab20darkblue, very thick]table [x=Step, y=r_msmarco_bm_5_5000_ff - accuracy, col sep=comma] {data/beam_path_inverse.csv};
        \addlegendentry{Width: 5}
        \addplot[mark=none, tab20darkred, very thick] table [x=Step, y=r_msmarco_bm_1_5000_ff - accuracy, col sep=comma] {data/beam_path_inverse.csv};
        \addlegendentry{Width: 1}
      \end{axis}
    \end{tikzpicture}}
    \end{figure}
\end{frame}

\begin{frame}{Inference Time}
  \begin{figure}[ht]
    \begin{minipage}{.45\linewidth}
      \centering
	\resizebox{1.16\textwidth}{!}{%
  \begin{tikzpicture}
    \hspace{-1.0cm}
    \begin{axis}
      [	%legend pos=outer north east,
        legend cell align={left},
        legend style={legend pos=north west},
        xlabel=beam width,
        ylabel=seconds,
        %ylabel shift = 0.1 pt,
        scaled y ticks = false,
        grid=major,
        xtick={1, 10, 20, 30, 40, 50},
        xticklabels={1, 10, 20, 30, 40, 50},
        ytick={0, 30, 60, 90, 120, 150, 180},
        ytick={0, 30, 60, 90, 120, 150, 180},
        extra x ticks={25},
        extra x tick labels={\textbf{MS MARCO}},
        extra x tick style={grid=none,ticks=major,ticklabel pos=right},
        xmin=1,
        xmax=50,
        ymin=0,
        ymax=180,
        width=8cm,
        height=8cm]
      \addplot[mark=none, tab20darkred, very thick] table [x=width, y=numbers, col sep=comma] {data/length2_msmarco.csv};
      \addlegendentry{Length: 2}
      \addplot [mark=none, tab20darkblue, very thick]table [x=width, y=numbers, col sep=comma] {data/length3_msmarco.csv};
      \addlegendentry{Length: 3}
    \end{axis}
  \end{tikzpicture}}
    \end{minipage}
    \hspace*{0.5cm}
    \begin{minipage}{.45\linewidth}
      \centering
	\resizebox{1.16\textwidth}{!}{%
  \begin{tikzpicture}
    \begin{axis}
      [	%legend pos=outer north east,
        legend cell align={left},
        legend style={legend pos=north west},
        xlabel=beam width,
        ylabel=seconds,
        grid=major,
        xtick={1, 10, 20, 30, 40, 50},
        xticklabels={1, 10, 20, 30, 40, 50},
        ytick={0, 30, 60, 90, 120, 150, 180},
        ytick={0, 30, 60, 90, 120, 150, 180},
        extra x ticks={25},
        extra x tick labels={\textbf{SemEval}},
        extra x tick style={grid=none,ticks=major,ticklabel pos=right},
        xmin=1,
        xmax=50,
        ymin=0,
        ymax=180,
        width=8cm,
        height=8cm]
      \addplot[mark=none, tab20darkred, very thick] table [x=width, y=numbers, col sep=comma] {data/length2_semeval.csv};
      \addlegendentry{Length: 2}
      \addplot [mark=none, tab20darkblue, very thick]table [x=width, y=numbers, col sep=comma] {data/length3_semeval.csv};
      \addlegendentry{Length: 3}
    \end{axis}
  \end{tikzpicture}}
    \end{minipage}
  \end{figure}
\end{frame}

\begin{frame}{Inverse Edges}
  \begin{itemize}
    \item Pro:
    \begin{itemize}
      \item Undo wrong actions
      \item Reach nodes that could otherwise not be reached in $T$ steps
    \end{itemize}
    \item Con:
    \begin{itemize}
      \item Introduction of false positives
      \item Atleast in theory it might not make sense to take an inverse edge
    \end{itemize}
    \item Does $e_1$ cause $e_4$? and $T = 2$:
  \end{itemize}
   \tikzset{person/.style={draw,rectangle,fill=tab20darkgreen!60,rounded corners,inner sep=5pt, minimum height=0.7cm} }
  \tikzset{object/.style={draw,rectangle,fill=tab20darkgreen!60,rounded corners, inner sep=5pt, minimum height=0.7cm}}
  \tikzset{type/.style={draw,rectangle,fill=lightgray,inner sep=5pt}}
  \tikzset{>=triangle 45}
  \tikzset{every picture/.style=thick}
  \tikzstyle{every node}=[font=\small]
    \tikzstyle{arrow} = [->, thick,>=stealth']

  %\begin{textblock*}{5cm}(9cm,1.8cm) 
  \begin{figure}
    \centering
  \resizebox{0.4\textwidth}{!}{
  \begin{tikzpicture}[remember picture]
    
    %\node[object, xshift=2cm,yshift=2.2cm, anchor=north east] (P1) {fatigue};
    \node[object, fill=tab20darkblue!60] (P1) {$e_1$};
    \node[object, right= 0.7cm of P1] (P2) {$e_2$};
    \node[object, right= 0.7cm of P2] (P3) {$e_3$};
    \node[object, right= 0.7cm of P3, fill=tab20darkblue!60] (P4) {$e_4$};
    \node[object, below= 0.7cm of P2, fill=tab20darkblue!60] (P5) {$e_5$};

    \draw[arrow, line width=1.2pt] (P1) -- (P2);
    \draw[arrow, line width=1.2pt] (P2) -- (P3);
    \draw[arrow, line width=1.2pt] (P3) -- (P4);
    \draw[arrow, line width=1.2pt] (P1) -- (P5);
    \draw[arrow, line width=1.2pt] (P4) -- (P5);

  \end{tikzpicture}}
\end{figure}
  %\end{textblock*}
\end{frame}

\begin{frame}{Examples}
  \begin{table}
    \centering
    \scriptsize
    \begin{tabular}{m{10cm} m{0.2cm}}
      \toprule
      \multicolumn{2}{c}{\textbf{Agent 3-Hop}} \\
      \midrule
       \vspace{0.2cm}
      \textbf{Cause:} h.\ pylori \hspace{0.54cm} \textbf{Effect:} vomiting &\\ 
       \vspace{0.2cm}
      \textbf{Path:} h.\ pylori $\xRightarrow{\text{cause}}$ peptic ulcer disease $\xRightarrow{\text{cause}}$ vomiting $\xRightarrow{\text{\textit{STAY}}}$ vomiting \vspace{0.2cm}\\
      \midrule
       \vspace{0.2cm}
      \textbf{Cause:} Xanax \hspace{0.92cm} \textbf{Effect:} hiccups &\\ 
       \vspace{0.2cm}
      \textbf{Path:} xanax $\xRightarrow{\text{cause}}$ anxiety $\xRightarrow{\text{cause}}$ stress $\xRightarrow{\text{cause}}$ hiccups \vspace{0.2cm}\\
      \midrule
       \vspace{0.2cm}
      \textbf{Cause:} chocolate \hspace{0.39cm} \textbf{Effect:} constipation &\\ 
       \vspace{0.2cm}
      \textbf{Path:} chocolate $\xRightarrow{\text{cause}}$ constipation $\xRightarrow{\text{cause}}$ depression $\xRightarrow{\text{cause}^{-1}}$ constipation \vspace{0.2cm}\\
      \midrule
       \vspace{0.2cm}
      \textbf{Cause:} rainfall \hspace{0.80cm} \textbf{Effect:} flooding &\\ 
       \vspace{0.2cm}
      \textbf{Path:} rainfall $\xRightarrow{\text{cause}}$ flooding $\xRightarrow{\text{cause}}$ landslides $\xRightarrow{\text{\textit{STAY}}}$ landslides \vspace{0.2cm}\\
      \bottomrule
    \end{tabular}
    \end{table}
\end{frame}

\end{document}